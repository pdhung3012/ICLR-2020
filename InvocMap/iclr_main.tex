
\documentclass{article} % For LaTeX2e
\usepackage{iclr2020_conference,times}

% Optional math commands from https://github.com/goodfeli/dlbook_notation.
\input{math_commands.tex}

\usepackage{hyperref}
\usepackage{url}


\title{InvocMap: Mapping Method Names to Method Invocations via Machine Learning}

% Authors must not appear in the submitted version. They should be hidden
% as long as the \iclrfinalcopy macro remains commented out below.
% Non-anonymous submissions will be rejected without review.


% The \author macro works with any number of authors. There are two commands
% used to separate the names and addresses of multiple authors: \And and \AND.
%
% Using \And between authors leaves it to \LaTeX{} to determine where to break
% the lines. Using \AND forces a linebreak at that point. So, if \LaTeX{}
% puts 3 of 4 authors names on the first line, and the last on the second
% line, try using \AND instead of \And before the third author name.

\newcommand{\fix}{\marginpar{FIX}}
\newcommand{\new}{\marginpar{NEW}}

%\iclrfinalcopy % Uncomment for camera-ready version, but NOT for submission.
\begin{document}


\maketitle

\begin{abstract}
	Implementing correct  method invocation is an important task for software developers. However, this is a challenging work, since the structure of method invocation can be complicated. In this work, we propose InvocMap, a code completion tool that allows developers to get the correct implementation structure of multiple method invocations from a list of method names inside code context. InvocMap can predict the nested method invocations which their names didn't appear in the list of input method names given by developers. To achieve this, we analyze the Method Invocations by four levels of abstraction. We build a Machine Translation engine to learn the mapping from first level to the third level of abstraction of multiple method invocations, which only requires developers to manually add local variables from generated expression to get the final code. We evaluate our proposed approach on six popular libraries: JDK, Android, GWT, Joda-Time, Hibernate, and Xstream. With the training corpus of 2.86 million method invocations extracted from 1000 Java Github projects and the testing corpus extracted from 120 online forums code snippets, InvocMap  achieves accuracy up to 84 in F1-score depending on how much information of context provided along with method names, which shows its potential for auto code completion.    
\end{abstract}
\input{introduction.tex}


\bibliography{iclr2020_conference}
\bibliographystyle{iclr2020_conference}



\end{document}
