\section{Discussion}
There are some threats to validity of our work. First, the data might not be representative. To alleviate this threat, we select high quality Github projects for training by selecting highest stars project which uses six popular libraries and select reliable online forums to get the code corpus.  Second, we evaluate the accuracy of the translation system automatically. We consider the input for translation as all method names and evaluate the accuracy of their respected ASTs. In reality, developers might write full implementation of MI if (s)he remembered how to implement that function along with writing method names in case they need suggestion. Thus, the method names of already implemented MIs can be skipped if we consider about the accuracy of the inference of method names that developers need the suggestion. We will hire people to set up this type of evaluation in future.\\
There are a few challenges that we need to improve for the translation. First, we still require developers to remember the method names for the inference. This could be a challenges for developers at a very beginning level of expertise. We consider that the more appropriate input can be Natural Language (NL) text descriptions by developers instead of method names and local entities. An interesting future work should be extending the original SMT model to accept natural language description by processing the description to have the most appropriate method names and local entities automatically for the input of method name to AST translation.  Second, since in reality developers can write the full MIs along with method names, InvocMap still doesn't take advantage of information provided by implemented MIs. This brings us to improve our work by a specific Machine Translation research for the case that some of source tokens we already know the result, and how we use those information to increase the accuracy, which can be useful  in applying SMT in other areas. Third, our SMT model still not be able to generate the MI at the fully details level. We will combine SMT with program analysis for the inference from level 1 to level 4 of MIs in future works. 
\section{Related Works}
Machine Learning has been applied widely in Software Engineering applications \cite{Allamanis:2018:SML:3236632.3212695}. Generating code by machine learning is an interesting but also confront challenges. There is a research by \cite{DBLP:journals/corr/BaroneS17} shows that the inference of code from documentation by machine translation achieved very low accuracy results on both SMT and Neural Machine Translation (NMT) models learned from practical large scale code corpus. There are two reasons cause this challenges. First, large scale code corpus contains noise data \cite{Pascarella:2017:CCC:3104188.3104217}. Second, the structure of AST Node is complicate for a machine translation system to learn about the syntactically correct of generated code as shown in \cite{DBLP:journals/corr/BaroneS17}. \cite{Gu:2016:DAL:2950290.2950334} propose an approach to achieve the implementation from  in natural language description. However, the output of their tool consists only sequence of APIs which is in level 2 of our abstraction for MIs. In our work, we target the inference of MI in level 3 with the ability of complex AST structure of MIs.
\\
There are several other inputs to get the complete code in other researches. \cite{7372046} derive the code in C\# language from code in Java language by machine translation. \cite{Gvero:2015:SJE:2814270.2814295,Gu:2016:DAL:2950290.2950334} generate the code from natural language descriptions. In these works, they consider the textual description as the full information for the inference. We consider our code generation problem in a different angle, which we take advantage of the surrounding context along with the textual description of method name in our work. \cite{6227236} propose a graph based code completion tool that suggest the full code snippet when developers are writing an incomplete code. This work focuses on completing the code from a part of the code. We propose an inference from the skeleton of method invocations, which is in form of sequence of method names, to the implementation of method invocations.
\section*{Conclusion}
In this work, we proposed InvocMap, a SMT engine for inferring
the ASTs of method invocations from a list of method
names and code context. By the evaluation on corpus
collected from Github projects and online forums, we demonstrated
the potential of our approach for auto code completion. A major
advantage of InvocMap is that it is built on the idea of abstracting
method invocations by four different levels. We provided an algorithm
to achieve AST of method invocations for the method invocations inference. As
future works, we will work on extending the SMT model to
support inputs from multiple natural language descriptions of
multiple method invocations, along with investigation of machine learning techniques for improving the accuracy. 




