\section{Related Works}
Machine Learning has been applied widely in Software Engineering applications \cite{Allamanis:2018:SML:3236632.3212695}. Generating code by machine learning is an interesting but also confront challenges. There is a research by \cite{DBLP:journals/corr/BaroneS17} shows that the inference of code from documentation by machine translation achieved very low accuracy results on both SMT and Neural Machine Translation (NMT) models learned from practical large scale code corpus. There are two reasons cause to this challenge. First, large scale code corpus contains noise data \cite{Pascarella:2017:CCC:3104188.3104217}. Second, the structure of AST Node is complicate for a machine translation system to learn about the syntactically correct of generated code as shown in \cite{DBLP:journals/corr/BaroneS17}. \cite{Gu:2016:DAL:2950290.2950334} propose an approach to achieve the implementation from  in natural language description. However, the output of their tool consists only sequence of APIs which is in level 2 of our abstraction for MIs. In our work, we target the inference of MI in level 3 with the ability of complex AST structure of MIs.
\\
There are several other inputs to get the complete code in other researches. \cite{7372046} derive the code in C\# language from code in Java language by machine translation. \cite{Gvero:2015:SJE:2814270.2814295,Gu:2016:DAL:2950290.2950334} generate the code from natural language descriptions. In these works, they consider the textual description as the full information for the inference. We consider our code generation problem in a different angle, which we take advantage of the surrounding context along with the textual description of method name in our work. \cite{6227236} propose a graph based code completion tool that suggest the full code snippet when developers are writing an incomplete code. This work focuses on completing the code from a part of the code. We propose an inference from the skeleton of method invocations, which is in form of sequence of method names, to the implementation of method invocations.
\section*{Conclusion}
In this work, we proposed InvocMap, a SMT engine for inferring
the ASTs of method invocations from a list of method
names and code context. By the evaluation on corpus
collected from Github projects and online forums, we demonstrated
the potential of our approach for auto code completion. A major
advantage of InvocMap is that it is built on the idea of abstracting
method invocations by four different levels. We provided an algorithm
to achieve AST of method invocations for the method invocations inference. As
future works, we will work on extending the SMT model to
support inputs from multiple natural language descriptions of
multiple method invocations, along with investigation of machine learning techniques for improving the accuracy. 




