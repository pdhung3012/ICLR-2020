We summarize the engines inside MethodInfoToCode on Figure \ref{fig:ApproachOverview}. From developer view, MethodInfoToCode provides a plugin inside with Java code editor to allow (s)he to write a single method name of a list method names inside the code environment. From this input, Method will translate each incomplete method names to respective ASTs. These ASTs reflect the complex structure of Method Invocation which might be inconvenient for developers to remembers. They were abstracted at level 3 in our definition, means they only require developers to add local variables, local methods or literals to get the final code. We will discuss about MI in level 3 in the next section. The ability of inferring ASTs for code completion relies on the Statistical Translation module from a PBMT system. The training process is done by the Statistical Learning module. This module learned from the data extracted from large scale Github code corpus  \cite{id:Github}. This data contains information about how developers implement ASTs for MIs as the Target Data. More important, it stores the mapping between the association between the list of method names and code context to the list of ASTs and code context which were developed by the community of Github users. In general, our statistical approach takes advantages of the knowledge of implementing MIs of experience developers, representing it by a machine learning model to help non-experience developers in retrieving good implementation of MIs.
\\
The underlying idea of MethodInfoToCode is that in this work, we consider the translation solution as the self learning approach from the incomplete representation of code to the more complete representation of code. In the other words, the source language is the incomplete representation while the target language is the complete representation of code. The fundamental type of tokens in our translation system is the method name in the source language and its implementation as an AST in the target language. We abstract the information about AST to token by the Method Invocation Abstractor module. This module will visit each Method Invocations from training, representing MIs with structure in level 3 of completeness for the target language and categorizing all MIs with the same information at this level with the same unique identification in the corpus.
\\
Besides Method Invocation, we also represent in the source and target language information about the context for each MIs. We consider context information as the tokens extracted from other types of AST Nodes defined by JDT \cite{id:ASTDocumentation}. The source language contains information about the name of each element inside a specific AST node, while the target language has information about the types of each elements. To do that, we define a list of rules for each types of AST nodes which we discuss in later section. The implementation of extracting other information beside MIs is done by the Context Extractor module. The Context Extractor and Method Invocation Abstractor contribute as components for the Training and Testing AST Visitor. These two visitor works based on the extension of well-known ASTVisitor pattern provided by JDT \cite{id:ASTDocumentation}, which traverse each types of AST node to get tokens. While the Train AST Visitor extracts both the source and target tokens, the Test AST Visitor extracts information of the source side, including method names inputted by developers.




related work
Machine Learning has been applied widely in Software Engineering applications \cite{Allamanis:2018:SML:3236632.3212695}. \cite{DBLP:journals/corr/abs-1803-09473} represents source code by a neural network model and applies it for method name recommendation. \cite{DBLP:journals/corr/abs-1812-07170,DBLP:journals/corr/abs-1901-09102} focus on apply the neural machine translation for learning and infer the change, including bugging change inside the code environment. \cite{8453132,Hellendoorn:2018:DLT:3236024.3236051} propose a solution for type inference from incomplete code for Java and Python based on Statistical Machine Translation and Neural Machine Translation. Another interesting research problem that can apply machine learning is to generate documentation from code environment.  \cite{Oda:2015:LGP:2916135.2916173} optimize the original SMT for pseudo code generation. \cite{Phan:2017:SLI:3102962.3102971} apply the SMT for inferring documentation from implementation of behavior exception inside code. In general, SMT and Neural Machine Translation (NMT) are applied and optimized for various inference problems to provide utilities for developers.
\\